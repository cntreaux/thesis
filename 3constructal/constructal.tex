\subsection{Constructal Design}

``The pyramid and the quarry grow at the same time. If the pyramid is a positive architecture (y > 0), the quarry is its negative. Such positive-negative pairs are everywhere in history and geography, even though modern advances in transportation technology tend to obscure them "

``Pyramids and ant hills are like the dried beds of rivers, cracked mud, and dendritic crystals (snowflakes): They are traces (fossils) of the optimized flow configurations that once existed. The universal phenomenon is the generation of flow architecture, and the principle is the constructal law: \textbf{\underline{For a flow system to persist in time (to survive), its configuration must change such that it provides easier and easier access to its currents.}} [author's emphasis] In a flow system, easier access means less thermodynamic imperfection (friction, flow resistances, drops, shocks) for what flows through the river basin or the animal. The optimal distribution of these numerous and highly diverse imperfections is the flow architecture itself (lung, river basin, blood vascularization, atmospheric circulation, etc.). . . In the making of a pyramid, the constructal law calls for the expenditure of minimum work. This principle delivers the \textit{location} and \textit{shape} of the edifice.''

In the same way, rammed-earth structures are traces of animal collectives that have known to expend work to converge a particular soil gradation from local soilscape to site (area to point), compose a workable mixture

\section{INTRODUCTION}

\begin{flushright}
  \small{
  \textit{``The history of building construction can be construed as a narrative of the inertia and momentum of two divergent construction logistics. One mode[, discussed above,] has very minimal historical inertia coupled with great current industrial momentum (the muli-layered assemblies of modernity.) The other has great historical, physical, and thermodynamic inertia that is coupled with minimal industrial momentum in the contemporary building industry/building science industry (more monolithic assemblies and masses). The former follows the short history of the twentieth century ``rationalization" of construction, air-conditioning, factory production, lightweight envelopes, and, more recently, mass customization. The latter is a several-thousand-year history of accumulative knowledge and performance all but forgotten in the interesting yet hubristically selective amnesia of twentieth century architecture."}}\\ --- Kiel Moe. \\ \textit{Convergence}. 2013.
\end{flushright}

``Rammed-earth"/``pis\'e de terre"/``tapia"/``h\=angt\v u" refers to an earthen building material formed by a particular mechanical process. Observably, rammed-earth forms are appearing around the U.S. with a breadth and depth previously unseen. They are visible east and west of the Rocky Mountains predominantly around academia, professional design, and building science/industry. Rammed-earth has archaeological, anonymous, and autochthonous roots in ancient China, Africa, the Middle East, Old Europe, and other regions globally \cite{RAMMEDEARTHHOUSE}.

\vspace{5mm}

\noindent The tectonic of rammed-earth within an image of the contemporary U.S. building culture is the subject of this thesis. The object of concern is the model for rammed-earth conceptualization and construction that has emerged since the latter part of the nineteenth century, vis-\`a-vis rammed-earth's traditional construction logistic.

\begin{flushright}
\small{
\textit{``[T]he culture that once was slow-moving, and allowed ample time for adaptation, now changes so rapidly that adaptation cannot keep up with it. No sooner is adjustment of one kind begun than the culture takes a further turn and forces the adjustment in a new direction. No adjustment is ever finished. And the essential condition on the process --- that it should in fact have time to reach its equilibrium --- is violated. This has all actually happened. In our own civilization, the process of adaptation and selection which we have seen at work in the unselfconscious cultures has plainly disappeared."}}\\ --- Christopher Alexander. \\ \textit{Notes on the Synthesis of Form}. 1964.
\end{flushright}

\clearpage

\subsection{The Model is the Message}

Marshall McLuhan noted that (\textit{Understanding Media}, 1964), with respect to media/technology, the medium is the message. The rammed-earth building medium \textit{qua} traditional rammed-earth conveys a perennial socio-technical desire for a durable, economical, and sustainable form of building. This is not an unseen desire in U.S. building history. In the late-eighteenth century, French architect Fran\c cois Cointereaux advocated rammed-earth architecture (residential and agricultural) to Thomas Jefferson to suit America's burgeoning rural economy; painted as ``the economical building art of the ancients, perfected and made more universal\footnote{\url{http://archive.is/yWexi}}." During the Great Depression, rammed-earth briefly held the attention of the New Deal-era Resettlement Administration as an economical building alternative. During the 1960s and 1970s, rammed-earth attracted marginal interest from the environmental movements following the global recognition of adverse anthropogenic effects on the biosphere \cite{GARDENDALE}.

It is argued here that around this point in time (60s, 70s), the Second Law finally began to have its way with rammed-earth's inertial, coherent, and adaptable construction logistic. The environmentally-aware milieu raised rammed-earth into a technologically dependent world caught between (in binary terms) developmentalism and zero-growth\footnote{\url{http://archive.is/sOf7w}}. The United Nations Department of Economic and Social Affairs (1964) output a disordering of rammed-earth called soil-cement.

\begin{flushright}
\small{
\textit{``Provided that natural soil possesses a combination of certain characteristics, however, it can be subjected to the process known as `stabilization'. The effect of adding a stabilizing agent like Portland cement, for instance, is not ony to enhance its best qualities but to impart to it other properties which soil alone does not possess."}}\\ --- Augusto A. Enteiche G.  Alexander. \\ \textit{Soil-Cement: Its Use in Building}. 1964.
\end{flushright}


this model continues to contemporary rammed-earth building to a point where rammed earth is nearly synonymous with csre


\begin{flushright}
\small{
\textit{
``Contemporary stabilized rammed earth (SRE) draws upon traditional rammed earth (RE) methods and materials, often incorporating reinforcing steel and rigid insulation, enhancing the structural and energy performance of the walls while satisfying building codes. SRE structures are typically engineered by licensed Structural Engineers using the Concrete Building Code or the Masonry Building Code."}} \\ --- Bly Windstorm and Arno Schmidt. \\ \textit{A Report of Contemporary Rammed-Earth Construction and Research in North America}. 2013.
\end{flushright}

medium is still the message, csre conveys sustainability at a convenience. sterling's amplified version: the model is the message. relevant for a world of building spimes. preserve sustainability with a model for rammed-earth that bridges tradition with the magic of instantaneous communication and computation.

\subsection{How Much Does Your Building Weigh?}

 massiveness: opportunity and challenge: transportation and stabilization

 emergy and intelligent standardization

 \begin{flushright}
 \small{
 \textit{
 ``Sustainability is never a static goal. It can only be a pro-
 cess. Previous ideas about “sustainability” are not and will never be tenable. A small, beautiful, modest, hand-crafted society, living in harmony with its eco-region, relentlessly parsimonious in its use of energy and resources, can't learn enough about itself to survive. In its bucolic quietude, it may appear timeless, but the clock is ticking for it as it does for all societies. It can avoid many conventional threats by adburing large-scale, clumsy technologies, but modesty does not make one invisible. That society isn’t keeping track---in its loathing for industria

 ism, it forfeits far too much command-and-control over its physical circumstances. Its bliss is ignorance

 A truly sustainable society has to be sustainable enough to prevail against the unforeseen. The unfore- seen, by definition, can’t be outplanned. This implies that serendipity is necessary. We can’t know what we need to know; so there need to be large stores of unplanned knowledge."}} \\ --- Bly Windstorm and Arno Schmidt. \\ \textit{A Report of Contemporary Rammed-Earth Construction and Research in North America}. 2013.
 \end{flushright}









% Contemporary technologies such as cement-stabilized, pre-insulated, and pre-fabricated walls---innovations currently driven by standard fixations on insulation, energy efficiency, and mass-production---disorder the coherence between the industrially-unprocessed material and its formative methodology. Apart from historic preservation, for which there is no positive heritage of rammed-earth building in America to preserve, it is hypothesized herein that this coherence of local material with localized building methodology is the key to rammed-earth's veritable sustainability (physically and conceptually enduring for millenia, while concurrently sustaining its respective inhabitants).

\section{INTRODUCTION}


``Rammed-earth"/``pis\'e de terre"/``tapia"/``h\=angt\v u" refers to an earthen building material formed by a particular technical process. This simple synchrony of building material and building method has propagated to countless building cultures from its pre-ancient origins in Asia, Africa, the Middle East, and other geographic locations \cite{RAMMEDEARTHHOUSE}. Perhaps more interesting than the independent conception of rammed-earth building across continents is the fact that it has remained virtually unchanged until relatively recently.

\begin{flushright}
\small{
\textit{``Have we not in Africa and in Spain walls of earth, known as `formacean' walls? From the fact that they are moulded, rather than built, by enclosing earth within a frame of boards, constructed on either side. These walls will last for centuries, are proof against rain, wind, and fire, and are superior in solidity to any cement."}}\\ --- Pliny the Elder. \\ \textit{The Natural History}. c.a. 77 A.D.
\end{flushright}

Rammed-earth exists traditionally as a monolithic matrix of graded soil particles sourced locally and solidified through simple tools and work. It is currently resurging in contemporary building-related sectors for its minimal geo-ecological interference and intrinsic hygrothermal properties suitable for energy-conscious living. In one word, rammed-earth is gaining traction in the U.S. as a \underline{sustainable} way of building. However, just as the ancients were not building with anthropogenic effects on the biosphere in mind, how does our ecologically/energetically-crisised building culture approach rammed-earth's historical ability to sustain through the occlusion of our own technological dependence and scientific rigor?

\begin{flushright}
\small{
\textit{``[T]he culture that once was slow-moving, and allowed ample time for adaptation, now changes so rapidly that adaptation cannot keep up with it. No sooner is adjustment of one kind begun than the culture takes a further turn and forces the adjustment in a new direction. No adjustment is ever finished. And the essential condition on the process --- that it should in fact have time to reach its equilibrium --- is violated. This has all actually happened. In our own civilization, the process of adaptation and selection which we have seen at work in the unselfconscious cultures has plainly disappeared."}}\\ --- Christopher Alexander. \\ \textit{Notes on the Synthesis of Form}. 1964.
\end{flushright}

Until building codes (U.B.C., I.B.C.) generally assumed authority over building, rammed-earth sustained its own specifically generic material composition and construction logistic, facilitating adaptation by various socio-technical cultures throughout the past millennia. ``Specifically generic" is a term attributed to Kiel Moe describing a program for building, not necessarily a style, that organizes the elements of building towards the contemporarily missing sense of adaptability Alexander mentions in the quote above \cite{MOECONVERGENCE}. Stable rammed-earth building forms could be achieved through a set of design and construction heuristics, not prescriptions, readily adapting to site and purpose.

For instance, rammed-earth has been a vernacular form of architecture in rural China, it was presented to then-Secretary of State Thomas Jefferson by French architect Fran\c cois Cointereaux as ``the economical building art of the ancients, perfected and made more universal\footnote{\url{http://archive.is/yWexi}}," it briefly held the attention of the New Deal-era Resettlement Administration as an economical building alternative during the Great Depression \cite{GARDENDALE}, and it captured marginal interest during the environmental movements of the 1960s and 1970s \cite{GARDENDALE}. Critically, conceptual and technical developments from the traditional rammed-earth constitution drive a modernized rammed-earth industry and normalize similarly modernized rammed-earth homes, some valued in the multimillion-dollar range\footnote{\url{http://archive.is/K853p}}. As rammed-earth science/innovation/marketing/design/construction et cetera approach what could be called the mainstream in the United States, the Second Law is finally beginning to have its way with rammed-earth at its conceptual and technical cores.

\begin{flushright}
\small{
\textit{
``Contemporary stabilized rammed earth (SRE) draws upon traditional rammed earth (RE) methods and materials, often incorporating reinforcing steel and rigid insulation, enhancing the structural and energy performance of the walls while satisfying building codes. SRE structures are typically engineered by licensed Structural Engineers using the Concrete Building Code or the Masonry Building Code."}} \\ --- Bly Windstorm and Arno Schmidt. \\ \textit{A Report of Contemporary Rammed-Earth Construction and Research in North America}. 2013.
\end{flushright}

Contemporary technologies such as cement-stabilized, pre-insulated, and pre-fabricated walls---innovations currently driven by standard fixations on insulation, energy efficiency, and mass-production---disorder the coherence between the industrially-unprocessed material and its formative methodology. Apart from historic preservation, for which there is no positive heritage of rammed-earth building in America to preserve, it is hypothesized herein that this coherence of local material with localized building methodology is the key to rammed-earth's veritable sustainability (physically and conceptually enduring for millenia, while concurrently sustaining its respective inhabitants).

\begin{flushright}
  \small{
  \textit{``The history of building construction can be construed as a narrative of the inertia and momentum of two divergent construction logistics. One mode, discussed above, has very minimal historical inertia coupled with great current industrial momentum (the muli-layered assemblies of modernity.) The other has great historical, physical, and thermodynamic inertia that is coupled with minimal industrial momentum in the contemporary building industry/building science industry (more monolithic assemblies and masses). The former follows the short history of the twentieth century ``rationalization" of construction, air-conditioning, factory production, lightweight envelopes, and, more recently, mass customization. The latter is a several-thousand-year history of accumulative knowledge and performance all but forgotten in the interesting yet hubristically selective amnesia of twentiety century architecture."}}\\ --- Kiel Moe. \\ \textit{Convergence}. 2013.
\end{flushright}

What follows is an attempt to rationalize and design a system of design that bridges rammed-earth's historical ability to sustain itself, its inhabitants, and the biosphere, with the contemporary mega-concept of sustainable development and mega-structure of building itself. Thermodynamics and computation are regarded as principal fields in which rammed-earth may be understood materially and methodologically, pre-architecturally and post-scientifically. Specifically, thermodynamics accounts for the physical flows of energy/matter involved in rammed-earth forms from the scale of physiological dynamics to the scale of the biosphere as a planetary solar engine. Computation accounts for the collective analyses and syntheses of information necessary to realize rammed-earth structures in a scalable and secure manner.

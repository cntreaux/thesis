\section{INTRODUCTION}

\begin{flushright}
  \small{
  \textit{``The history of building construction can be construed as a narrative of the inertia and momentum of two divergent construction logistics. One mode[, discussed above,] has very minimal historical inertia coupled with great current industrial momentum (the muli-layered assemblies of modernity.) The other has great historical, physical, and thermodynamic inertia that is coupled with minimal industrial momentum in the contemporary building industry/building science industry (more monolithic assemblies and masses). The former follows the short history of the twentieth century ``rationalization" of construction, air-conditioning, factory production, lightweight envelopes, and, more recently, mass customization. The latter is a several-thousand-year history of accumulative knowledge and performance all but forgotten in the interesting yet hubristically selective amnesia of twentieth century architecture."}}\\ --- Kiel Moe. \\ \textit{Convergence}. 2013.
\end{flushright}

``Rammed-earth" refers to an earthen building material formed by a particular mechanical process. Rammed-earth architecture has ancient archaeological, anonymous, and autochthonous roots in China, Africa, Europe, India, the Middle East, and other regions globally \cite{CHRONO}. Observably, rammed-earth forms have been (re)appearing in the U.S. over the past half-century with a frequency and technical gain atypical of earthen construction, especially in the West. Contemporary rammed-earth forms are predominantly connected to academia, professional architectural design, and building science/industry.

Rammed-earth structures have appeared relatively recently on the campuses of M.I.T. (2005), Stanford (2015), and Princeton (2016)\footnote{\url{https://archive.is/5gPbZ} (M.I.T.); \url{https://archive.is/VhpW2} (Stanford); \url{https://archive.is/9SF6K} (Princeton)}. Rammed Earth Works (the designers of Stanford's Windhover Contemplative Center) is one of multiple professional architectural firms to be designing modernized rammed-earth buildings and installations valued in the multimillion-dollar range\footnote{\url{https://archive.is/K853p}}. David Easton, the founder of Rammed Earth Works (1976), also invented PISE (Pneumatically Impacted Rammed Earth) and Watershed Blocks (under a \$750k grant from the N.S.F.); a sprayed application of wet soil-cement and a mechanical system for mass-producing modular soil-cement blocks, respectively. SIREWALL (Structural Insulated Rammed-Earth, B.C., Canada) is a rammed-earth-based building product incorporating an intermediary layer of patented insulation\footnote{\url{https://archive.is/Sf9fu} (PISE); \url{https://archive.is/x3iwg} (Watershed Materials); \url{https://archive.is/s0cHI} (SIREWALL)}. Numerous technical papers concerning rammed-earth's structural and thermal properties have been published globally. A select few include: \textit{Analysis of the hygrothermal functional properties of stabilised rammed-earth materials} by Hall and Allinson. (2009), \textit{Modeling rammed earth wall using discrete element method} by Bui et al. (2015), and \textit{Measured and simulated thermal behaviour in rammed earth houses in a hot-arid climate. Part B: Comfort} by Beckett et al. (2017).

\vspace{5mm}

\underline{Hypothesis:} Along the decades of rammed-earth's previous resurgence in the U.S., at the eco-movements of the 60s and 70s, rammed-earth took on a greater socio-technical momentum while standards, practices, and policies of the building culture remained inert. The result was a form-based function of modern construction---a virtual image of sustainability---reducing, reusing, and recycling a historically function-based form of complexity, persistence, and adaptability. The rammed-earth material \textit{and} method morphed more significantly in forty years than it had in four thousand.

If Doctor E. is right, and we can not solve problems by using the same kind of thinking we used when we created them, then it stands to reason that greater control of contemporary rammed-earth technology is not as much a determinant of sustainability as the model by which rammed-earth forms are conceptualized, codified, communicated, designed, logisticized, and constructed.

\vspace{5mm}

\underline{Objective:} \textbf{Design a system} of design (a coalescing model of models) drawing from contemporary theories and technology, reaching towards the principles and heuristics of the traditional rammed-earth material and method. The model is directed towards a computational system (digital accumulator and distributor of information) capable of organizing two flows:

1. The flow of soil types from deposit to building site (determinants of rammed-earth's embodied energy and building performance).

2. The flow of knowledge between builders (determinant of rammed-earth's standardization and design).


\begin{flushright}
\small{
\textit{``[T]he culture that once was slow-moving, and allowed ample time for adaptation, now changes so rapidly that adaptation cannot keep up with it. No sooner is adjustment of one kind begun than the culture takes a further turn and forces the adjustment in a new direction. No adjustment is ever finished. And the essential condition on the process --- that it should in fact have time to reach its equilibrium --- is violated. This has all actually happened. In our own civilization, the process of adaptation and selection which we have seen at work in the unselfconscious cultures has plainly disappeared."}}\\ --- Christopher Alexander. \\ \textit{Notes on the Synthesis of Form}. 1964.
\end{flushright}


% A precedent similar in its regard for the past is the \textit{Manifesto for responsible architecture}, a one-page document assembled by a collective of architectural collectives and presented at the U.N. Climate Change Conference\in Paris (2015)\footnote{\url{https://web.archive.org/web/20180429205023/https://www.ace-cae.eu/fileadmin/New_Upload/8._Images/News/2015/Manifesto_EN_2.pdf}}. Allegedly, the Manifesto derived from the fossil of Caral, Peru (3,000-1800 B.C.), during a retrospective on the ``high engineering" involved in its structure, from ductwork to urban planning\footnote{\url{http://archive.is/e3wBp}}.



\clearpage

\subsection{The Model is the Message}

Marshall McLuhan noted (\textit{Understanding Media}, 1964) that, with respect to media/technology, the medium is the message. The rammed-earth building medium \textit{qua} pre-modern rammed-earth conveys a natural socio-technical desire for a functional, durable, and economical form of building. Without any positive rammed-earth heritage to draw from, the U.S. has nonetheless considered or adopted rammed-earth construction during energy-sensitive phrases of its history. For instance, in the late-eighteenth century, French architect/builder Fran\c cois Cointereaux presented Thomas Jefferson and America's burgeoning rural economy with a case for rammed-earth architecture. Encoded in a copy of \textit{Ecole d'architecture rurale} (Paris, 1790-91), Cointereaux believed that if America adopted ``the economical building art of the ancients, perfected and made more universal," She would incur a great physiocratic power. Jefferson reacted indifferently, in a letter to William Short, ``how far it may offer benefit here superior to the methods of the country, founded in the actual circumstances of the country as to the combined costs of labour \& materials, and the circumstances of durability comfort \& appearance, must be the result of calculation."\footnote{\url{http://archive.is/yWexi} (Cointereaux to Jefferson); \url{http://archive.is/ozqQv} (Jefferson to Short)}

Rammed-earth later appeared in \textit{Popular Mechanics (Vol. 41, No. 2, 1924)} and \textit{The Farmers Bulletin  (No. 1500, 1926)}, endorsed as a frugal, Do-It-Yourself building method. During the Great Depression, rammed-earth briefly held the attention of the New Deal-era Resettlement Administration as an economical building alternative fit for an over-abundance of available labor. Around the 1960s and 1970s, rammed-earth attracted marginal interest from the environmental movements, following the global recognition of troubling anthropogenic effects on the biosphere and building's major role aside this phenomenon \cite{GARDENDALE}. Speculatively, following this last wave of rammed-earth building in the U.S., although the material would continue in some form, the D.I.Y.ness was lost to the momentum/inertia of building practices.

In a gross linearization of history, it would appear that a growing field around ``ecodevelopment" in the 1960s/1970s effectively reintroduced rammed-earth building for the first time into a technologically dependent world aware of the consequences of (in binary terms) developmentalist and zero-growth economic strategies\footnote{\url{http://archive.is/sOf7w}}. Hypothetically, at this shearing of [simply] technological positivity and technological negativity, building societies retrofit rammed-earth into a model through which cake could be had and also eaten. On one hand, the tried method would remain a symbol of building sustainability. On the other, seemingly innocuous technical changes to rammed-earth's composition and construction would modernize the material-method, ensuring marketability, scalability, standardization, and security.

Rammed-earth v2.0 manifested as ``soil-cement", also known as ``cement stabilized rammed-earth." Over time, it was layered with modernizing assemblages of mechanization, pre-fabrication, insulation, transportation, seal-ification, svelte-ification and modularization. An early code of practice was \textit{Soil-Cement: Its Use in Building}, distributed by the U.N. Department of Economic and Social Affairs in 1964.

\begin{flushright}
\small{
\textit{``The use of simple compacted soil (natural earth) as a building material dates from time immemorial, and it can be said that ever since, and down to the present day, the method of building houses with earth has been used, because of its constructive qualities. Yet, despite its} good insulating and resistant properties [author's emphasis], \textit{there are limitations to the use of earth owing to its lack of strength and its vulnerability to moisture and the erosive effects of external agents. Provided that natural soil possesses a combination of certain characteristics, however, it can be subjected to the process known as `stabilization'. The effect of adding a stabilizing agent like Portland cement, for instance, is not only to enhance its best qualities but to impart to it other properties which soil alone does not possess."}}\\ --- Augusto A. Enteiche \\ \textit{Soil-Cement: Its Use in Building}. 1964.
\end{flushright}


\begin{flushright}
\small{
\textit{
``Contemporary stabilized rammed earth (SRE) draws upon traditional rammed earth (RE) methods and materials, often incorporating reinforcing steel and rigid insulation, enhancing the structural and energy performance of the walls while satisfying building codes. SRE structures are typically engineered by licensed Structural Engineers using the Concrete Building Code or the Masonry Building Code."}} \\ --- Bly Windstorm and Arno Schmidt. \\ \textit{A Report of Contemporary Rammed-Earth Construction and Research in North America}. 2013.
\end{flushright}


Generally, the above quotes represent modern/contemporary models concerning rammed-earth's function as a building material, physically and also with regard to their respective building cultures. Endemic to both rationalizations is [what is now seen to be \cite{MOECONVERGENCE}] a destructive reduction of vital qualities (or non-qualities) of rammed-earth building, e.g. ``resistant properties" and questionably ``[enhanced]" energy performance.

McLuhan's wisdom remains, the cement-stabilized rammed-earth medium conveys the message that, as Bruce Sterling noted (\textit{Shaping Things}, 2005), the model is the message. Rammed-earth ceases to exist as an unselfconscious technology of sustenance and increasingly becomes defined by complicated logistics, stylistic preferences, and virtual references. This is to say that contemporary rammed-earth \textit{in rem} does not necessarily possess the property of sustainability. Instead, the historical ability of rammed-earth to sustain itself, its settlers, its ecosystem, and the biosphere at large is a phenomenon emerging from the inanimate and animate collectives involved in rammed-earth building, i.e. its model.

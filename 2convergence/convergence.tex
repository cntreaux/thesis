
Architect/builder/author/professor Kiel Moe has authored a number of texts and a number of buildings in and around the past decade that cogently embody novel theories about building(s) and energy. \textit{Convergence} is one such textual work based around the notion that ``matter is but captured energy" \cite{MOECONVERGENCE}. This realization, and associated theorems, is central to the model for contemporary rammed-earth building at hand.

As Professor Moe explains much more thoroughly and lucidly in \textit{Insulating Modernism}, disparate fields related to building(s) have inevitably imparted their own characters onto modern building(s), leading to questionable building practices from a standpoint of ``building energy". Industrial engineering, as a prime example, translated applications of nineteenth century thermodynamics (e.g. artificial refrigeration through systems closed by insulation) to building-scale, setting in motion a contra-rational model [if air is primarily an insulator, why temper spaces with convective heat transfer?] for H.V.A.C. systems still in effect today. Following from this professional paradigm is a cultural language fixated on ``energy-efficiency" [transformations of energy always operate at 100\% efficiency] and ``energy-conservation" [equating exergy exclusively to The Grid], when nature (and rammed-earth building) offers a more nuanced potential for engaging energy/matter systems \textit{MOEIM}.

Aside from the fact that rammed-earth building is primarily an architectural endeavor, a contemporary architectural theory motivates historically-oriented rammed-earth building because it presupposes a model learned from the iatrogenic effects of twentieth century modernism and anticipates twenty-first century non-equilibrium thermodynamics. Not accidentally, the theory mirrors incipient rammed-earth building principles manifesting before science or architecture were known as such.

Professor Moe explicitly references rammed-earth at least twice. Once, in the Building Lecture Series at the University of Virginia\footnote{\url{http://archive.is/u9TKf}}, in the context of rammed-earth as a thermally massive building material. Capillary to this vein, the material quantities thermal effusivity (\textit{e}) and thermal diffusivity ($\alpha$) contribute to a more effective understanding of built environments as thermally transient, interactive, radiant, qualitative systems rather than scientifically ideal systems forever operating in the steady-state mode. Second, Professor Moe references rammed-earth as a case study in \textit{Convergence}: the Granturismo Earth and Stone project in southern Portugal. Initially a reforestation initiative funded by the European Union, the project entailed ten rammed-earth and stone structures in the inner Algarve region suited for tourism and recreation. In this remote area, the locally-sourced property of rammed-earth proved to be critical for design, construction, economic, and ecological reasons. Furthermore, the Algarve does possess a positive heritage in rammed-earth building and Granturismo was an opportunity to ``[make] the history of the Algarve material culture apparent while [the material selection reinvests] in the labor and skill connected to that material." \cite{MOECONVERGENCE}

The aforementioned objectives, organization of soil flow from deposit to site, and conceptual flow between builders, may be construed as a consequence of the convergence of matter and energy. The first flow is a matter of emergy, that rammed-earth does not materialize for free. The second flow is a matter of exergy, that knowledge is one of the most potent forms of exergy, and predicates the contemporary development of rammed-earth energy systems.

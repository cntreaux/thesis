
Architect/builder/author/professor Kiel Moe has authored a number of texts and a number of buildings in and around the past decade that cogently embody a novel theory about building(s) and energy. ``An Architectural Agenda for Energy", the subtitle of \textit{Convergence}, describes a ``more totalizing" conjunction of building systems with the subtle yet opportunistic complexities of energy typically siloed off to engineering and applied science. In this way, the Agenda predicates healthier building dynamics (more durable and effective flows of matter/energy), in turn, predicating healthier inhabitants (re: Sick Building Syndrome, Building Related Illness) and healthier ecosystems (holistic environmental accounting) \cite{MOECONVERGENCE}. The Agenda and its network of references form an ideal system with which rammed-earth's contemporary presence may be bridged to its sustainable history. This section is dedicated to fitting rammed-earth design and building further into the Agenda, hopefully with minimal gain in conceptual entropy from Professor Moe's work. At once, the Agenda reflects latent, useful properties of the rammed-earth material and method as well as the terrifically complicated, complicating mega-structure of building rammed-earth currently finds itself in.

Professor Moe explicitly references rammed-earth at least twice. Once, in the Building Lecture Series at the University of Virginia\footnote{\url{http://archive.is/u9TKf}}, in the context of rammed-earth as a thermally massive building material. Capillary to this vein, Professor Moe discusses the material quantities thermal effusivity (\textit{e}) and thermal diffusivity ($\alpha$) contributing to a more wholesome understanding of building materials as thermally transient, interactive, qualitative systems rather than scientifically ideal systems forever operating in the steady-state mode. Second, Professor Moe references rammed-earth as a case study in \textit{Convergence}: the Granturismo Earth and Stone project in southern Portugal. Initially a reforestation initiative funded by the European Union, the project entailed ten rammed-earth and stone structures in the inner Algarve region suited for tourism and recreation. In this remote area, the locally-sourced property of rammed-earth proved to be critical for design, construction, economic, and ecological reasons. Furthermore, the Algarve does possess a positive heritage in rammed-earth building and Granturismo was an opportunity to ``[make] the history of the Algarve material culture apparent while [the material selection reinvests] in the labor and skill connected to that material." \cite{MOECONVERGENCE}

In the next three subsections, rammed-earth codes, performance, and energetics are drawn around the Architectural Agenda for Energy. The ultimate motivation for doing so is founded in the fact that the Agenda values the historical inertia of rammed-earth with its contemporary potential in a profoundly integrated manner. Ostensibly, no other theory of building draws across disciplinary boundaries (engineering, physics, architecture, ecology, history, economics. . .) nor across scales (``From the Molecular to the Territorial"\footnote{\url{http://archive.is/7lXIB}}) so applicably.

At a high level, the socio-technical, building codification is regarded as a great challenge and a great opportunity to preserve traditional rammed-earth. At an intermediate level, material and thermal performances of rammed-earth are considered in light of standard building envelopes. At a third level, the quantum and the global simultaneously, quantities and qualities of energy are considered as they pertain to the design and construction of rammed-earth buildings.



\subsubsection{Rammed-Earth Codes}



\subsubsection{Rammed-Earth Performance}



\subsubsection{Rammed-Earth Energetics}
